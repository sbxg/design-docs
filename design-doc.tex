\documentclass{article}
\usepackage[T1]{fontenc}
\usepackage[utf8]{inputenc}
\usepackage[english]{babel}

\usepackage{hyperref}

\begin{document}

\section{Introduction}

SBXG is a build orchestrator that provides means to generate from sources or
foreign binaries a ready-to-flash SD card image for embedded systems.

SBXG orchestrates several components:
\begin{itemize}
\item a toolchain to compile most of the other components;
\item a bootloader builder;
\item a primary kernel builder;
\item an initramfs builder for the kernel that may rely on it;
\item a primary rootfs generator;
\item and a filesystem utility to generate the final image which will conain the
  desired components.
\end{itemize}

All of these components can be used individually, or chained together to
generate more complex systems. A brief description of each of them follow.

\subsection{Toolchain Retriever}

The \emph{toolchain} is a suite of software tools that allows to create another
software. When we refer to a toolchain, we refer to a (cross) compiler driver
and its own components (like gcc and the collection programs it relies on).

Since SBXG is primarily targetting embedded platforms, the toolchain will mostly
be arm-flavoured. The different arm architectures will vary in function of the
\emph{processor} embedded on the SoC that ships with an \emph{embedded board}.

We select our toolchains from \url{https://www.linaro.org/}, but custom ones can
of course be provided.


\subsection{Bootloader Builder}
\subsection{Kernel Builder}
\subsection{Initramfs Builder}
\subsection{Rootfs Generator}

\end{document}
